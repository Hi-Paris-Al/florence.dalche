Let us recall the expression of the pinball loss \seep{figure:pinball}:
\begin{dmath}[compact]
    v_{\hyperparameter}: (y,\,y') \in
    \reals^2 \mapsto  \max{(\hyperparameter (y-y'), (\hyperparameter
    -1)(y-y'))} \in \reals.
\end{dmath}
\begin{floatingfigure}{.7\textwidth}
    \centering
    \begin{tikzpicture}
    \begin{axis}[%
        axis x line = center,
        axis y line = center,
        disabledatascaling,
        axis equal image,
        xlabel = {$y - h(x)$},
        ylabel = {$v_{\theta}(y, h(x))$},
        x label style={at={(axis description cs:1.0, 0.0)}, anchor=north},
        ticks = none
        ]
        \addplot[orange, domain=-1:1, samples at={-1, 0, 1}]
            {abs(.8 - (x < 0)) * abs(x)};
        \draw
            (-0.5, 0.1) coordinate (a)
            (0, 0) coordinate (b)
            (-0.5, 0) coordinate (c);
        \draw
            pic["$\theta - 1$", draw=black, -,dashed, angle eccentricity=1.2,
                angle radius=2.3cm]
            {angle=a--b--c};
        \draw
            (0.5, 0.4) coordinate (d)
            (0, 0) coordinate (e)
            (0.5, 0) coordinate (f);
        \draw
            pic["$\theta$", draw=black, -,dashed, angle eccentricity=1.2,
                angle radius=2.3cm]
            {angle=f--e--d};
    \end{axis}
\end{tikzpicture}

    \caption{\label{figure:pinball} Pinball loss for $\hyperparameter=0.8$.}
\end{floatingfigure}
\begin{proposition}\label{proposition:generalized_excess_risk}
    Let $X,Y$ be two \acp{rv} respectively taking values in $\mcX$ and
    $\reals$, and $q \colon \mcX \to \mathcal{F}([0,1],\mathbb{R})$ the
    associated conditional quantile function.  Let $\mu$ be a positive measure
    on $[0,1]$ such that $ \int_{0}^1 \expectation\left[
    v_{\hyperparameter}\left(Y,q(X)(\hyperparameter)\right)\right] \mathrm{d}
    \mu(\hyperparameter) < \infty$.  Then for $\forall h \in
    \functionspace{\mathcal{X}}{\functionspace{\closedinterval{0}{1}}{\reals}}$
    \begin{dmath*}
        \risk(h) - \risk(q) \geq 0,
    \end{dmath*}
    where $R$ is the risk defined in \cref{equation:h-objective}.
\end{proposition}
\begin{proof}
    The proof is based on the one given in \citep{li2007quantile} for a single
    quantile. Let $f \in
    \functionspace{\mcX}{\functionspace{\closedinterval{0}{1}}{\reals}}$,
    $\hyperparameter \in (0,1)$ and $(x,y) \in \mcX \times \reals$. Let also
    \begin{align*}
        s &=
        \begin{cases}
            1 ~ \text{if } y \leq f(x)(\hyperparameter)      \\
            0  \text{ otherwise}
        \end{cases},&
        t &=
        \begin{cases}
            1 ~ \text{if } y \leq q(x)(\hyperparameter)      \\
            0  \text{ otherwise}
        \end{cases}.
    \end{align*}
    It holds that
    \begin{dmath*}
        v_{\hyperparameter}(y,h(x)(\hyperparameter)) -
        v_{\hyperparameter}(y,q(x)(\hyperparameter))
        = \hyperparameter(1-s)(y-h(x)(\hyperparameter)) + (\hyperparameter -
        1)s(y-h(x)(\hyperparameter)) -
        \hyperparameter(1-t)(y -q(x)(\hyperparameter)) -
        (\hyperparameter-1)t(y-q(x)(\hyperparameter))
        = \hyperparameter(1-t)(q(x)(\hyperparameter) - h(x)(\hyperparameter)) +
        \hyperparameter((1-t)-(1-s))h(x)(\hyperparameter) +
        (\hyperparameter-1)t(q(x)(\hyperparameter - h(x)(\hyperparameter))) +
        (\hyperparameter-1)(t-s)h(x)(\hyperparameter) + (t-s)y\nonumber
        =     (\hyperparameter -t)(q(x)(\hyperparameter) -
        h(x)(\hyperparameter)) +
        (t-s)(y-h(x)(\hyperparameter)).\label{eq:decompo_pinball}
    \end{dmath*}
    Then, notice that
    \begin{dmath*}[compact]
        \expectation{[(\hyperparameter - t)(q(X)(\hyperparameter) -
        h(X)(\hyperparameter))]} =
        \expectation{[\expectation{[(\hyperparameter - t)(q(X)(\hyperparameter)
        - h(X)(\hyperparameter))]} | X]} =
        \expectation{[\expectation{[(\hyperparameter - t) | X ]}
        (q(X)(\hyperparameter) - h(X)(\hyperparameter))]}
    \end{dmath*}
    and since $q$ is the true quantile function,
    \begin{align*}
        \expectation{ [t | X]} = \expectation{ [\mathbf{1}_{\{ Y \leq
        q(X)(\hyperparameter)\}} | X]} = \probability{[Y \leq
        q(X)(\hyperparameter) | X]} = \hyperparameter,
    \end{align*}
    so
    \begin{align*}
        \expectation{[(\hyperparameter - t)(q(X)(\hyperparameter) -
        h(X)(\hyperparameter))]} =0.
    \end{align*}
    Moreover, $(t-s)$ is negative when $q(x)(\hyperparameter) \leq y \leq
    h(x)(\hyperparameter)$, positive when $h(x)(\hyperparameter) \leq y \leq
    q(x)(\hyperparameter)$ and $0$ otherwise, thus the quantity
    $(t-s)(y-h(x)(\hyperparameter))$ is always positive. As a consequence,
    \begin{dmath*}[compact]
        R(h) - R(q) = \int_{[0,1]}
        \expectation{[v_{\hyperparameter}(Y,h(X)(\hyperparameter)) -
        v_{\hyperparameter}(Y,q(X)(\hyperparameter))]} \mathrm{d}
        \mu(\hyperparameter) \geq 0
    \end{dmath*}
    which concludes the proof.
\end{proof}
The \cref{proposition:generalized_excess_risk} allows us to derive conditions
under which the minimization of the risk above yields the true quantile
function. Under the assumption that (i) $q$ is continuous (as seen as a
function of two variables), (ii) $\mathrm{Supp}(\mu) = [0,1] $, then the
minimization of the integrated pinball loss performed in the space of
continuous functions yields the true quantile function on the support of
$\probability_{X,Y}$.
